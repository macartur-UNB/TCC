\part{Conclusão}

\chapter[Considerações Finais]{Considerações Finais}

\section{Conclusões desta primeira etapa do trabalho}

Nesta primeira etapa do trabalho, analisando os 3 primeiros
 métodos de redução do tempo, é possível perceber que todos
 eles reduzem o tempo de compilação.

Para o método de Guardas de Inclusão deve-se evitar a
 utilização de pragma once , uma vez que este aumenta
 o tempo de compilação em quase 50\%. A melhor guarda
 de inclusão é a por redundância, como evidenciado por
 John Lakos. No entanto, nos experimentos aqui realizados,
 esta diferença foi pouco visível.

Para o método de Forward declaration foi confirmado que
 este reduz o tempo de compilação, com no máximo 5\% de redução
 dentre os projetos analisados. No entanto podem ser observados
 ganhos maiores a depender do número de arquivos modificados
 entre cada compilação, uma vez que a forward declaration
 reduz a dependência entre os  arquivos.

\section{Expectativas para a continuação}

Para a segunda parte do trabalho é esperado que os
 outros métodos e ferramentas resultem também em
 reduções do tempo de compilação. Tais ganhos
 também serão investigados experimentalmente.  

\section{Potenciais riscos ao trabalhos}

Para a segunda parte do trabalho há o risco de não 
serem encontrados projetos adequados para a aplicação
 dos métodos restantes.

