\part{Conclusão}

\chapter[Conclusão]{Conclusão}

    Neste trabalho foi apresentado 5 métodos de redução do tempo de compilação: guardas de inclusão,
 declaração implicita de estrutura, ponteiro privado de implementação, uso de flags de compilação para 
 otimização de baíxo nível, flags de processamento paralelo, e o uso de 2 ferramentas \textit{linker gold} e o
 \textit{ccache}.

    Dentre as técnicas utilizadas as tecnicas que envolvem alteração de código não apresentaram
 valores satisfatórios na coleta de dados realizada. Ao contrario destes os métodos que apresentam
 parametrização e ferramentas obtiveram um resultado bastante satisfatório, na qual é possível
 reduzir mais de 50\% do tempo de compilação de um projeto.

    O método que apresentou melhores resultados utilizou as \textit{flags} de processamento paralelo, 
 na qual proporcionou uma redução minima de 45\% do tempo de compilação do projeto. Dentre as ferramentas
 utilizadas a ferramenta \textit{ccache} é a mais eficiente pois reduziu mais de 50\% do tempo de compilação
 de todos os projetos apresentados.

    Para a coleta de dados foi utilizado 3 ambientes diferentes, na qual utilizando virtualização foram instalados
 o sistema operacional Windows 7, Mac OS Yosemite e Ubuntu. Dentre estes ambientes o Windows 7, que utilizou o \textit{cygwin}
 como ambiente de desenvolvimento,  foi o ambiente que resultou no 
 maior tempo de compilação dos projetos sem a utilização de nenhuma técnica ou ferramenta. O ambiente Mac OS Yosemite
 foi o ambiente rápido dentre os selecionados pois este utiliza o compilador com \textit{back-end} LLVM.

    Tendo estas informações este trabalho conseguiu atingir os seu objetivo geral, que foi conseguir
 selecionar métodos e ferramentas que consegui-se atingir uma redução em mais de 30\% do tempo de compilação
 de projetos de código aberto selecionados.
