\part{Conclusão}

\chapter[Considerações Finais]{Considerações Finais}

\section{Trabalhos Futuros}
	Como trabalho futuro é possível tentar verificar o comportamento do método de implementação
 privada de classe, utilizando recompilações com alterações em partes de um projeto e sem
 realizar remoção dos arquivos objetos já gerados.
	Outra abordagem que pode ser explorada é realizar um estudo das ferramentas de redução do
 tempo de compilação, uma vez que neste trabalho a ferramenta foi utilizada com as configurações
 padrão da instalação da ferramenta sendo possível aumentar a quantidade de cache
 da ferramenta ccache.

\section{Conclusões}

	Durante este trabalho foram pesquisadas e analisadas métodos e ferramentas que reduzem
 o tempo de compilação de projetos escritos em C++. Foram elas Guardas de inclusão,
 declaração implicita de estrutura e implementação privada de uma classe,  utilização das
 ferramentas gold e ccache e acionamento de flags de otimização providas pelos
 compiladores de c++ padrão dos 3 diferentes sistemas operacionais e flags o controle de
 flags de otimização de código do compilador.

	Os três sistemas operacionais foram Windows 7, Mac OS Yosemite e Ubuntu(Linux). O ambiente
 linux com a utilização dos métodos e ferramentas de compilação foi o ambiente com maior
 tempo de compilação, utilizando o ambiente de desenvolvimento (cygwin). O melhores tempos de
 compilação utilizando os métodos e as ferramentas aqui analisados foram encontrados no
 Mac OS Yosemite devido a utilização do compilador padrão clang++ que foi
 contruído com o objetivo de ser mais rápido que o compilador padrão do gnu o g++, que vem
 como default no Ubuntu.

    Com a coleta dos dados e analise de cada estudo de caso nos projetos aqui analisados resultou que
 os métodos de alteração de código não foi obtido valores satisfatórios pois a redução do
 tempo de compilação. Para as flags de processamento paralelo utilizando o programa make
 obteve resultados resultados consideraveis uma vez que com elas é possivel reduzir mais
 de 30\% do tempo de compilação de um projeto. Analisando as ferramentas o programa gold
 não foi encontrado uma redução satisfatória, uma vez que a redução é muito baixo, ao
 contrário da ferramenta ccache que conseguiu produzir resultados em que reduziu mais
 de 50\% do tempo de compilação de um projeto.

	Após os estudos o objetivo principal destes trabalho foi alcançado pois foram
 analisadas as ferramentas aqui propostas em projetos de software livres escritos em C++ 
em diferentes ambientes e analisados e selecionados os melhores métodos dentre os que possuem
um impacto de no mínimo 30\% na redução dos projetos.
