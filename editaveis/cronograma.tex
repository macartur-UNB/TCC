\chapter[Cronograma]{Cronograma}

\section{Atividades já realizadas}

Para a primeira etapa do TCC foram realizados experimentos de 3 métodos,
 Este momento inicial permitiu a realização de um estudo mas aprofundado
 sobre o assunto e identificar a viabilidade do projeto.


\subsection{Atividades realizadas}

\begin{enumerate}
	\item		estudo bibliográfico sobre o tema;
	\item		experimento de utilização de guarda de inclusão em diferentes contextos , e qual a mais eficiente;
	\item		experimento de utilização do Makefile para rodar tarefas em paralelo.
	\item		experimento de utilização de forward declaration para a redução do tempo de compilação de um software livre;
	\item		analise dos resultados e escrita do TCC1.
\end{enumerate}

\begin{table}[h]
\centering
\begin{tabular}{|l|l|l|l|l|}
Atividade & Março & Abril & Maio & Junho \\ \hline
1         & X     & X     &      &       \\ \hline
2         &       & X     &      &       \\ \hline
3         & X     &       & X    &       \\ \hline
4         &       &       & X    &       \\ \hline
5         &       &       & X    & X     \\ \hline
\end{tabular} 
\caption{Cronograma Etapa 1}
\label{cronograma1}
\end{table}

\subsection {Atividades a serem realizadas}

Para a segunda etapa do TCC, será realizado as seguintes atividades:

\begin{itemize}
	\item experimento com utilização de Pimpl Idiom (Ponteiro opaco).
    \item utilização de ferramentas que podem auxiliar no processo de compilação e recompilação de um projeto, como por exemplo ccache, gold e wrap.
    \item estudo da melhor forma de se aplicar a utilização de headers pré-compilados;
    \item experimentos e estudo sobre a melhor forma de aplicar otimização de baixo nível de um compilador em códigos de C++;
    \item analise dos resultados dos experimentos e escrita do TCC2 , contendo um tópico com as melhores práticas de utilização das ferramentas e métodos analisados.
\end{itemize}

\begin{table}[h]
\centering
\begin{tabular}{|l|l|l|l|l|l|}
Atividade & Junho & Agosto & Setembro & Outubro& Novembro \\ \hline
1         & X     & X      &          &        &     \\ \hline
2         &       & X      & X        &        &     \\  \hline
3         &       &        & X        &        &   \\ \hline
4         &       &        &          & X      &  \\ \hline
5         &       &        & X        & X      & X \\   \hline
\end{tabular} 
\caption{Cronograma Etapa 2}
\label{cronograma2}
\end{table}
