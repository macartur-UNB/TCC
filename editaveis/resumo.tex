\begin{resumo}
Neste trabalho serão analisados métodos e ferramentas
 que proporcionem redução do tempo de compilação
 de programas escritos em C++, a fim de obter respostas mais
 rápidas às modificações no código e reduzir os gastos com recursos
 (humanos, de hardware e de tempo) utilizados no processo
 de compilação. Assim este trabalho consiste em uma
 análise preliminar de métodos e ferramentas que possam
 ser utilizadas em conjunto com o compilador g++ 
 nesta redução do tempo de compilação. Os métodos utilizados
 neste trabalho preliminar foram as guardas de inclusão,
 a declaração implícita de estruturas (\textit{forward declaration}),
 e processamento paralelo através do comando make.
 Os métodos analisados resultaram em uma redução média 
 do tempo de compilação em torno de 20\% e 40\%.

 \vspace{\onelineskip}
    
 \noindent
 \textbf{Palavras-chaves}: gcc. g++. c++. c. hpp. h.hxx. compilador. montador. gerador de código intermediário. forward declaration. declaração implícita.  guardas de inclusão externa. guardas de inclusão externa. redundância de guardas de inclusão. include. preprocessador. projeto. programa. pragma once. desenvolvedor. minutos. segundos. horas. semanas. compilação. código intermediário. desenvolvimento. equipes. empresas. fases da compilação. just in time. máquina virtual. ferramentas. metodologia. cronograma.

\end{resumo}
