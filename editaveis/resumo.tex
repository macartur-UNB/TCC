\begin{resumo}
Neste trabalho serão analisados métodos e ferramentas
 que proporcionem redução do tempo de compilação
 de programas escritos em C++, a fim de obter respostas mais
 rápidas às modificações no código e reduzir os gastos com recursos
 (humanos, de hardware e de tempo) utilizados no processo
 de compilação. Assim este trabalho consiste em uma
 análise preliminar de métodos e ferramentas que possam
 ser utilizadas em conjunto com o compilador g++ 
 nesta redução do tempo de compilação. Os métodos utilizados
 neste trabalho preliminar foram as guardas de inclusão,
 a declaração implícita de estruturas (\textit{forward declaration}),
 e processamento paralelo através do comando make.
 Os métodos analisados resultaram em uma redução média 
 do tempo de compilação em torno de 20\% e 40\%.

 \vspace{\onelineskip}
    
 \noindent
 \textbf{Palavras-chaves}: compilação. C++. g++. Makefile. pré-processador. 

\end{resumo}
