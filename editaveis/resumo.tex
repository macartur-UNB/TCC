\begin{resumo}
Neste trabalho serão analisados métodos e ferramentas
 que proporcionam redução do tempo de compilação
 de programas escritos em C++, a fim de obter respostas mais
 rápidas às modificações no código e reduzir os gastos com recursos
 (humanos, de hardware e de tempo) utilizados no processo
 de compilação. Assim este trabalho consiste em uma análise de
 métodos e ferramentas que possa ser utilizado com o compilador
  na redução do tempo de compilação. Os métodos utilizados neste
 trabalho foram guardas de inclusão, declaração implícita de estruturas,
 (\textit{forward declaration}), processamento paralelo com a ferramenta make,
 controle da ativação de flags de otimização, e ferramentas de auxílio
 a compilação(gold e ccache). Estes estudos foram aplicados em 3
 sistemas operacionais distintos.
 \vspace{\onelineskip}
    
 \noindent
 \textbf{Palavras-chaves}: compilação. C++. g++. Makefile. diretivas.

\end{resumo}
