\begin{resumo}
Neste trabalho serão analisados métodos e ferramentas
 que proporcionem redução do tempo de compilação
 de programas em C++, afim de obter respostas mais
 rápidas a modificações no código, desalocar recursos
 (humanos, de hardware e de tempo) utilizados no processo
 de compilação. Assim este trabalho consistem em uma
 analise preliminar de métodos e ferramentas que possa
 ser utilizada junto com o compilador g++ afim de proporcionar
 a redução do tempo de compilação. Alguns métodos utilizados
 neste trabalho preliminar foram:  guardas de inclusão,
 declaração implicita de estruturas(Forward Declaration),
 e processamento paralelo com makefile.
 Estes métodos analisados produziram em média uma redução
 do tempo de compilação entre 20\% e 40\%,.

 \vspace{\onelineskip}
    
 \noindent
 \textbf{Palavras-chaves}: gcc. g++. c++. c. hpp. h.hxx. compilador. montador. gerador de código intermediário. forward declaration. declaração implicita.  guardas de inclusão externa. guardas de inclusão externa. redundancia de guardas de inclusão. include. pré-processador. projeto. programa. pragma once. desenvolvedor. minutos. segundos. horas. semanas. compilação. código intermediário. desenvolvimento. equipes. empresas. fases da compilação. just in time. máquina virtual. ferramentas. metodologia. cronograma.

\end{resumo}
