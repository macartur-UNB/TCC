\part{Introdução}

\chapter*[Introdução]{Introdução}

Na computação os primeiros computadores eletrônicos foram "engenhocas
 monstruosas", ocupavam varias salas, consumiam tanta energia quanto
 uma fábrica, e custavam, em 1940, milhares de dólares, com o poder
 computacional de uma calculadora moderna. Para os
 programadores que usavam estas máquinas a hora do computador era
 mais valiosas que a dele. E eles programavam em linguagem de
 máquina \cite[pag.5]{ref6}.

Programadores ou desenvolvedores são pessoas capazes de criar instruções
 que podem ser executados por um computador, escrevendo o código-fonte em
 alguma linguagem de programação. Finalizado o código-fonte, a construção
 de um programa que seja executado em um computador pode ser feita por um
 compilador, programa capaz de fazer a transformação de
 código-fonte escrito na linguagem de programação escolhida para uma linguagem
 entendível pelo computador, que é o código de máquina \cite[pag.5]{ref6}, a qual
 se trata de uma sequência de instruções que serão executadas  pelo
 processador. Esta transformação recebe o nome de compilação \cite[pag.1]{ref5}.

Projetos de médio e grande porte gastam recursos e tempo a cada processo
 de compilação, com perdas que dependem do hardware utilizado e do tamanho
 do projeto, sendo que a duração de um ciclo de compilação pode variar de
 alguns minutos até semanas \cite[pag.5]{ref6}. Neste contexto, este trabalho
 irá realizar um levantamento de métodos e ferramentas para redução do
 tempo de compilação, uma vez que esta redução pode significar ganhos
 significativos de recursos humanos, financeiros e no cronograma de
 desenvolvimento de equipes e empresas. 

%\section*{Justificativa}

%Como dito anteriormente, o tempo de compilação um projeto pode variar
% de minutos até horas. Isto pode impactar no desenvolvimento de um
% projeto, visto que a cada modificação é necessário  uma recompilação
% para a reconstrução dos artefatos do projeto. A redução do processo
% de compilação permite respostas mais rápidas para as alterações
% realizadas no projeto antes da compilação, além de desalocar
% recursos (humanos, de hardware e de tempo) utilizados para
% a compilação.

\section*{Objetivos do Trabalho}

\subsection*{Objetivo Geral}

Analisar um conjunto de boas praticas em C++ e ferramentas auxiliares
 para a redução do tempo de compilação de projetos que utilizam como
 padrão o compilador g++, avaliando o impacto de cada estratégia no
 tempo de compilação de cada projeto.

\subsection*{Objetivos Específicos}

Para conseguir atingir o objetivo geral do trabalho foram listados
 alguns objetivos específicos:


\begin{enumerate}
    \item Levantamento de  métodos em C++ garantir menor acoplamento,
 maior coesão e melhora na redução do tempo de compilação;
    \item Levantamento de ferramentas que podem auxiliar na compilação
 de projetos utilizando o compilador g++ e linguagem C++;
    \item Elencar projetos de código livre para análise do tempo de
 compilação;
    \item Realizar experimentos que permitam avaliar o impacto de cada
 estratégia (método, ferramenta ou ambos) no tempo de compilação de
 cada projeto.
\end{enumerate}

\subsection*{Delimitação do Escopo}

Para delimitar o escopo e abrangência deste trabalho, e tornar possível
 a análise de projetos reais, será utilizado como referência uma linguagem
 de programação compilada conhecida por ser utilizada em sistemas embarcados,
 sistemas operacionais, sistemas críticos e jogos, dentre outras aplicações,
 que é a C++, criada por Bjarne Stroustroup\cite{BjarneC++}.

\subsection*{Organização do Trabalho}

O Capítulo 2 tem por objetivo introduzir o leitor aos conceitos abordados
 neste trabalho, descrevendo a diferença entre uma linguagem compilada e
 interpretada, o modelo \textit{just-in-time}, máquinas virtuais, o processo de compilação
 e suas etapas e os métodos a serem trabalhados.

O Capítulo 3 descreve um fluxo de trabalho realizado para a primeira etapa
 deste trabalho, contendo projetos escolhidos, ferramentas utilizadas e
 condução dos experimentos.

O Capítulo 4 apresenta os resultados alcançados até o presente momento com
 os estudos dos 3 primeiros métodos descritos na metodologia. Os outros
 métodos serão analisados na próxima etapa deste trabalho.

O Capítulo 5 contém a descrição das atividades realizadas na primeira
 etapa deste trabalho e as que serão realizadas na segunda bem como a ordem
 de realização das atividades.

Por fim o Capítulo 6  faz as considerações finais do trabalho até o presente
 momento.
