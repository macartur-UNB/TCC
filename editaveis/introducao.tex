\part{Introdução}

\chapter*[Introdução]{Introdução}

Na computação, os primeiros computadores eletrônicos foram \lq\lq engenhocas monstruosas\rq\rq,
 ocupavam varias salas, consumiam tanta energia quanto
 uma fábrica, e custavam, em 1940, milhares de dólares, com o poder
 computacional de uma calculadora moderna. Para os
 programadores que usavam estas máquinas a hora do computador era
 mais valiosas que a dele. E eles programavam em linguagem de
 máquina \cite[pág.5]{ref6}.

Programadores ou desenvolvedores são pessoas capazes de criar instruções
 que podem ser executadas por um computador, escrevendo o código-fonte em
 alguma linguagem de programação. Finalizado o código-fonte, a construção
 de um programa que será executado em um computador pode ser feita por um
 compilador, programa capaz de fazer a transformação de
 código-fonte escrito na linguagem de programação escolhida para uma linguagem
 entendível pelo computador, que é o código de máquina \cite[pág.5]{ref6}, a qual
 se trata de uma sequência de instruções binárias.
 Esta transformação recebe o nome de compilação \cite[pág.1]{ref5}.

Projetos de médio e grande porte gastam recursos e tempo a cada processo
 de compilação, com perdas que dependem do hardware utilizado e do tamanho
 do projeto, sendo que a duração de um ciclo de compilação pode variar de
 alguns minutos até semanas \cite[pág.5]{ref6}. Neste contexto, este trabalho
 irá realizar um levantamento de métodos e ferramentas para redução do
 tempo de compilação, uma vez que esta redução pode significar ganhos
 significativos de recursos humanos, financeiros e no cronograma de
 desenvolvimento de equipes e empresas. 

\section*{Objetivos do Trabalho}

\subsection*{Objetivo Geral}

Analisar um conjunto de boas práticas e métodos para a redução do
 tempo de compilação de projetos escritos em C++ em diferentes ambientes,
 avaliando o impacto de cada estratégia.
 Após a analise serão selecionadas as melhores abordagens
 para os diferentes ambientes de desenvolvimento, tendo como meta conseguir uma
 redução mínima de 30\% no tempo de compilação dos projetos analisados. 

\subsection*{Objetivos Específicos}

Para conseguir atingir o objetivo geral do trabalho foram listados
 alguns objetivos específicos:

\begin{enumerate}
    \item Levantamento de métodos em C++ que levam a menor acoplamento,
 maior coesão e melhora na redução do tempo de compilação;
    \item Levantamento de ferramentas que podem auxiliar na compilação
 de projetos escritos em linguagem C++;
    \item Elencar projetos de código livre para análise do tempo de
 compilação;
    \item Aplicar cada método aqui trabalhado para avaliar o impacto de cada
 estratégia (método, ferramenta ou ambos) no tempo de compilação de
 cada projeto.
    \item Elencar as melhores abordagems para cada método analisado.
\end{enumerate}


\subsection*{Delimitação do Escopo}

Para delimitar o escopo e abrangência deste trabalho, e tornar possível
 a análise de projetos reais, será utilizado como referência uma linguagem
 de programação compilada conhecida por ser utilizada em sistemas embarcados,
 sistemas operacionais, sistemas críticos e jogos, dentre outras aplicações,
 que é a C++, criada por Bjarne Stroustroup \cite{BjarneC++}.

\subsection*{Organização do Trabalho}

O Capítulo 1 tem por objetivo introduzir o leitor nos conceitos abordados
 neste trabalho, descrevendo a diferença entre uma linguagem compilada e
 interpretada, o modelo \textit{just-in-time}, máquinas virtuais, o processo de compilação
 e suas etapas e os métodos a serem trabalhados.

O Capítulo 2 descreve um fluxo de trabalho realizado para a primeira etapa
 deste trabalho, contendo projetos escolhidos, ferramentas utilizadas e
 condução dos método aplicado.

O Capítulo 3 apresenta os resultados alcançados depois da aplicação dos métodos, coleta de dados,
 bem como uma analise e seleção dos métodos que obtiveram melhores resultados em cada ambiente.
 
Por fim o Capítulo 4 aprensenta as considerações finais e quais os trabalhos futuros
 poderiam ser realizados e que não foram abordados neste trabalho.
